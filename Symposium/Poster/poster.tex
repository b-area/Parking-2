%==============================================================================
%== template for LATEX poster =================================================
%==============================================================================
%
%--A0 beamer slide-------------------------------------------------------------
\documentclass[final]{beamer} % use beamer
\usepackage[orientation=landscape,
            size=a0,          % poster size
            scale=1.2         % font scale factor
           ]{beamerposter}    % beamer in poster size
%
%--some needed packages--------------------------------------------------------
\usepackage[american]{babel}  % language 
\usepackage[utf8]{inputenc}   % std linux encoding

%

% ===========================================================================
% ==  The poster style ======================================================
% ===========================================================================
\usetheme{SCI}    % our poster style
%--set colors for blocks (without frame)---------------------------------------
  \setbeamercolor{block title}{fg=ngreen,bg=white}
  \setbeamercolor{block body}{fg=black,bg=white}
%--set colors for alerted blocks (with frame)----------------------------------
%--textcolor = fg, backgroundcolor = bg, dblue is the jacobs blue
  \setbeamercolor{block alerted title}{fg=white,bg=dblue!70}%frame color
  \setbeamercolor{block alerted body}{fg=black,bg=dblue!10}%body color
%

% ==========================================================================
% ==Title, date and authors of the poster===================================
% ==========================================================================
\title{Applying Na\"{i}ve Bayes Classifier in the Natural Language Toolkit \\
to classify recent tweets.}
\author{Calvin Canada and Eddy C. Borera}
\institute[Jacobs University]
          {School of Computing and Informatics, Lipscomb University,
           One University Park Dr., Nashville, TN}
\date{\today}
%
%==some usefull qm commands====================================================
%  |x>
\newcommand{\ket}[1]{\left\vert#1\right\rangle}
%  <x|
\newcommand{\bra}[1]{\left\langle#1\right\vert}
%  <x|y>
\newcommand{\braket}[2]{\left< #1 \vphantom{#2}\,
                        \right\vert\left.\!\vphantom{#1} #2 \right>}
%  <x|a|y>
\newcommand{\sandwich}[3]{\left< #1 \vphantom{#2 #3} \right|
                          #2 \left|\vphantom{#1 #2} #3 \right>}
%  d/dt
\newcommand{\ddt}{\frac{d}{dt}}
%  D/Dx
\newcommand{\pdd}[1]{\frac{\partial}{\partial#1}}
%  |x|
\newcommand{\abs}[1]{\left\vert#1\right\vert}
%  k_{x}
\newcommand{\kv}[1]{\mathbf{k}_{#1}}

% ========================
% Handy commands (Eddy)
% ========================
\newcommand{\argmax}{\operatornamewithlimits{argmax}}

%==============================================================================
%==the poster content==========================================================
%==============================================================================
\begin{document}
%--the poster is one beamer frame, so we have to start with:

\begin{frame}[t]
%--to seperate the poster in columns we can use the columns environment
 \begin{columns}[t] % the [t] options aligns the columns content at the top
 
  % ----------------------------------------------------------
  % -- the left column----------------------------------------
  % ----------------------------------------------------------
  \begin{column}{0.28\paperwidth}% the right size for a 3-column layout     
  	
  	%-- abstract block ------------
   	\begin{alertblock}{Abstract}
   	Natural Language Processing has been widely used to extract 
   	meaningful information from human language that appears in phone 
   	dialogues, newspaper stories, emails, product manuals, etc... 
   	It has been successfully applied in word editors to check and correct 
   	spelling errors, in email applications to detect spams, and in spoken 
   	dialogue system to translate spoken words into texts and vice versa. 
   	In this research, we combine Natural Language Processing with machine 
   	learning techniques to classify tweets as either sad, happy, or neutral. 
   	The ability to classify tweets could be considered a useful resource for 
   	companies and organizations to track reactions of the mass or specific 
   	targeted consumers regarding their products or recently published statements. 
   	In our work, we designed feature functions for the Natural Language Toolkit 
   	framework, written in Python, and applied them to a Na\"{i}ve Bayes Classifier 
   	to classify recent tweets retrieved from the Twitterarchivist.com website. 
   	Randomly selected training data are fed in to the classifier before testing 
   	the overall classification model on randomly sampled test tweets. We analyze 
   	the classification accuracy results to determine if further modifications 
   	are needed in feature functions.
   \end{alertblock}
	

	\setbeamercolor{block alerted title}{fg=black,bg=nred}%frame color
   	\setbeamercolor{block alerted body}{fg=black,bg=white}%body color
   	\begin{alertblock}{Introduction}
    		Natural Language Toolkit (NLTK) is a platform that can be used to work 
    with data that deals with managing human language. NLTK contains many classifiers, 
    one of which is the Naive Bayes classifier which is the one that we have implemented.
    The Naive Bayes classifier is a classifier that was built using the Na\"{i}ve Bayes
    algorithm as the cornerstone method of classifying. NLTK has been described as a
    practical introduction to programming for language processing. Many have used this
    classifier to perform actions such as sort through Spam, translate languages, and
    construct sentences in a given language. We combined this wonderful tool with the
    Python programming language to enable us to interpret and classify human language
    statements using Twitter data. We start by collecting data which we then divide into a
    train set and a test set. The train set is used to train the classifier, while the
    test set is fed to the classifier after this process to determine the accuracy of the
    our classifier.
   	\end{alertblock}
    \vskip2ex





	% ---------------------------------------
  	% -------------- References -------------
  	% ---------------------------------------
   	\begin{block}{References}
    		\begin{thebibliography}{7}
     		{\small %small is better for refs
       			\bibitem{bposter}
       			\url{http://www-i6.informatik.rwth-aachen.de/
                   ~dreuw/latexbeamerposter.php}
       			\bibitem{beamerdoc}
       			\url{http://www.ctan.org/tex-archive/
                   macros/latex/contrib/beamer/doc/beameruserguide.pdf}
       			\bibitem{pgf}
       			\url{http://www.ctan.org/tex-archive/help/Catalogue/entries/pgf.html}
     		}
    		\end{thebibliography}
   \end{block}
   \vskip2ex   
   
  \end{column}
  
  
  % -------------------------------------------------------------
  % ---------- Introduction (One Big Right Column) --------------
  % -------------------------------------------------------------
  \begin{column}{0.60\paperwidth} %thats the big right column
   
   \begin{block}{Introduction}
    Natural Language Toolkit (NLTK) is a platform that can be used to work 
    with data that deals with managing human language. NLTK contains many classifiers, 
    one of which is the Naive Bayes classifier which is the one that we have implemented.
    The Naive Bayes classifier is a classifier that was built using the Na\"{i}ve Bayes
    algorithm as the cornerstone method of classifying. NLTK has been described as a
    practical introduction to programming for language processing. Many have used this
    classifier to perform actions such as sort through Spam, translate languages, and
    construct sentences in a given language. We combined this wonderful tool with the
    Python programming language to enable us to interpret and classify human language
    statements using Twitter data. We start by collecting data which we then divide into a
    train set and a test set. The train set is used to train the classifier, while the
    test set is fed to the classifier after this process to determine the accuracy of the
    our classifier.
	\end{block}
	\vskip2ex
   

  	%--begin new columns to get alignment on top------------------------------------
 	\begin{columns}[t,totalwidth=0.60\paperwidth]
          		
  		\begin{column}{0.28\paperwidth}
			%------red block--------------------------------------
   			\setbeamercolor{block alerted title}{fg=black,bg=nred}%frame color
   			\setbeamercolor{block alerted body}{fg=black,bg=white}%body color
   			\begin{alertblock}{nred}
    		
   			\end{alertblock}
			
			%--orange block----------------------------------------
   			\setbeamercolor{block alerted title}{fg=black,bg=norange}%frame color
   			\setbeamercolor{block alerted body}{fg=black,bg=white}%body color
   			\begin{alertblock}{Na\"{i}ve Bayes Classifier}
	 			\begin{definition}[NBC]
	   				 Na\"{i}ve Bayes Classifier (NBC) is a supervised learning technique 
	   				that is based on the Bayes' Theorem. A document is classified
	   				by taking the most probable class label given a set of feature 
	   				values (\emph{maximum likelihood}).
    				\end{definition}
    				
    				Given a set of feature values $\left(f_1, f_2, \ldots, f_n \right)$, 
				the likelihood of a document to be in  a class label $c$ is computed 
				by the following equation:
				\begin{eqnarray}
					Pr(c \mid f_1, \ldots , f_n) & = & \frac{Pr(f_1, \ldots , f_n \mid c)
					                           Pr(c)}{Pr(f_1, \ldots , f_n)}  \nonumber \\
					&  & \nonumber \\
					& = &  \frac{Pr(c) \prod_{i=1}^{n} Pr(f_i \mid c)}{Pr(f_1, \ldots ,f_n)} \nonumber \\
			       & = &  \frac{Pr(c) \left[Pr(f_1 \mid c) \times \ldots \times Pr(f_n \mid c) \right] }{Pr(f_1, \ldots ,f_n)} \nonumber 
				\end{eqnarray}
					
				Then, the most likely class is determined by the following equation:
				\begin{eqnarray}
					\argmax_{c} Pr(c \mid f_1, \ldots , f_n) & = &  \argmax_{c} \frac{Pr(c) \prod_{i=1}^{n} Pr(f_i \mid c)}{Pr(f_1, \ldots ,f_n)} \nonumber \\
					& = &  \argmax_{c} Pr(c) \prod_{i=1}^{n} Pr(f_i \mid c) \nonumber
				\end{eqnarray}
    				
   			\end{alertblock}
  	\end{column}
  
	%--right column-----------------------------------------------------------------
  	\begin{column}{0.28\paperwidth}
		%--yellow block-----------------------------------------------------------------
   		\setbeamercolor{block alerted title}{fg=black,bg=nyellow}%frame color
   		\setbeamercolor{block alerted body}{fg=black,bg=white}%body color
   		\begin{alertblock}{NLTK}
    				Natural Language Toolkit (NLTK) is a platform that can be used to work 
    with data that deals with managing human language. NLTK contains many classifiers, 
    one of which is the Naive Bayes classifier which is the one that we have implemented
   		\end{alertblock}
		
		% ------------ Experiments --------------------------
   		\setbeamercolor{block alerted title}{fg=black,bg=ngreen}%frame color
   		\setbeamercolor{block alerted body}{fg=black,bg=white}%body color
   		\begin{alertblock}{Experiments}
    			
   		\end{alertblock}
   		
   		% ------------ Results --------------------------
   		\setbeamercolor{block alerted title}{fg=black,bg=dpurple}%frame color
   		\setbeamercolor{block alerted body}{fg=black,bg=white}%body color
   		\begin{alertblock}{Results}
    			
   		\end{alertblock}
   		
   		
   		\setbeamercolor{block alerted title}{fg=black,bg=dblue}%frame color
   		\setbeamercolor{block alerted body}{fg=black,bg=white}%body color
   		
   	\end{column}
   
  \end{columns}
  \vskip3ex  
  
  \begin{alertblock}{Conclusion}
    As you can see it is possible to make your poster very colorfull. But in 
    most cases this will this will overload your poster. If you don't change 
    the color settings you will get the default look, which consits of some
    shades of the jacobs blue and some decent green highlights. These colors 
    where chosen carefully to keep a consistent look of the poster. The 
    \emph{cpbgposter} style is installed our office computers, so you should be
    able to compile this example out of the box with pdflatex. If you want to
    work on your computer make sure that you have a recent TeX distribution 
    (TeXlive 2008, Miktex) and download the beamerthemecpbgposter.sty file from
    our teamwork page and put it in your local TeX directory.

% guess what this command is god for!
    \makeruleinbox
% it works, but causes some underfull/overfull \hbox warnings
    \begin{center}
     {\huge\vskip-1ex
     {\color{nred}H}{\color{norange}a}{\color{nyellow}p}
     {\color{ngreen}p}{\color{dblue}y}\\
     \TeX'ing!}
    \end{center}
   \end{alertblock}
  
  \end{column}
 \end{columns}
\end{frame}
\end{document}